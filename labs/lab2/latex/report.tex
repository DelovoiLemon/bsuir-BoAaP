\input{../../preamble}
\usepackage{listings}
\usepackage{xcolor}

%New colors defined below
\definecolor{codegreen}{rgb}{0,0.6,0}
\definecolor{codegray}{rgb}{0.5,0.5,0.5}
\definecolor{codeblue}{rgb}{0.05, 0.95, 0.96}
\definecolor{backcolour}{rgb}{0.95,0.95,0.92}

\lstdefinestyle{mystyle} {
    backgroundcolor=\color{backcolour}, commentstyle=\color{codegreen},
    keywordstyle=\color{codeblue},
    numberstyle=\tiny\color{codegray},
    stringstyle=\color{codeblue},
    basicstyle=\ttfamily\footnotesize,
    breakatwhitespace=false,         
    breaklines=true,                   
    keepspaces=true,             
    showspaces=false,                
    showstringspaces=false,
    showtabs=false,                  
    tabsize=2
}

\lstset{style=mystyle}

\begin{document}

\begin{titlepage}
	\newpage \null
	\begin{center}
		Министерство образования Республики Беларусь \\[0.4cm] 

			Учреждение образования \\

			\MakeUppercase{БЕЛОРУССКИЙ ГОСУДАРСТВЕННЫЙ УНИВЕРСИТЕТ ИНФОРМАТИКИ И РАДИОЭЛЕКТРОНИКИ} \\[0.4cm]

		Кафедра электронных вычислительных машин \\ [4cm]

		Отчет по лабораторной работе №2 \\

Реализация разветвляющихся алгоритмов
 \\ [5cm]
		
		\noindent
		\parbox[t]{0.5\textwidth}{\raggedright
		Выполнил: \\
		студент 1 курса \\
		группы № 348602 \\
		Трошкин Дмитрий Сергеевич}\hfill
		\parbox[t]{0.5\textwidth}{\raggedleft
		Проверил: \\
		Матюшкин Светослав Иванович}%	
		\vfill

		{\normalsize Минск 2023}
	\end{center}
\end{titlepage}

\section{Реализация разветвляющихся алгоритмов}

\textbf{Цель работы}: изучить операции сравнения, логические операции, 
операторы передачи управления if, switch, break, научиться пользоваться
простейшими компонентами организации переключений (СheckBox, RadioGroup).
Написать и отладить программу с разветвлениями.

\subsection{Третий уровень сложности}

\subsubsection{Условие}

Для динамического двухмерного массива решить поставленную задачу,
алгоритм которой реализовать в виде отдельной функции. При вводе исходных
данных выполнить проверку ввода нечисловых значений.

11. Дана вещественная квадратная матрица порядка n. \\
Получить целочисленную квадратную матрицу, в которой элемент равен 1, \\
если соответствующий ему элемент исходной матрицы больше элемента,\\ 
расположенного на главной диагонали, и равен 0 в противном случае.\\


\subsubsection{Исходный код}
\lstinputlisting{../src/task3.c}

\subsubsection{Пример}
\lstinputlisting{examples/example3.txt}

\end{document}
