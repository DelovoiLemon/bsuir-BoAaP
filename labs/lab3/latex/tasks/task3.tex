Составить программу по заданию из п. 3.3.2, в которой нахождение суммы 
S(x) и функции Y(x) организовать в виде отдельных функций, причем расчет
функции выполнять не по заданному значению n, а до тех пор пока разница
очередного значения суммы не будет отличаться от значения функции на 
некоторую величину (погрешность) $\varepsilon$, равную, например, 0.001 (0.0001), 
т. е. до тех пор пока |S(x) – Y(x)| <= $\varepsilon$. Определить количество шагов
вычисления суммы, при которых был достигнут указанный результат.

11.
$ 
	\begin{cases}
	S(x) = \sum_{k = 0}^{n}{\frac{k^2 + 1}{k!}(x/2)^k}, \\
	Y(x) = (x^2 / 4 + x / 2 + 1)e^{x/2}
	\end{cases}
$
